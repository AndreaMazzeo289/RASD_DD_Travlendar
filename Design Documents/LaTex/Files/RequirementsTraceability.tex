\begin{itemize}
	\item \textbf{[R1] The user must be logged into the system to access application features.}\\
As explained in the IITP part, there is dependency between Travel and Appointment Manager and Account Manager: the account managing infrastructure is created first and travel/appointment data are acquired from the database.

\item \textbf{[R2] The user must be able to choose the option of creating a new appointment.}\\
Appointment Manager implements the functionality createAppointment(). See section~\ref{subsec:newAppointment}: Create new appointment.

\item \textbf{[R3] The user must be able to choose the option of editing a selected appointment.}\\
Appointment Manager implements the functionality editAppointment(). See section~\ref{subsec:editAppointment}: Edit existing appointment.

\item \textbf{[R4] The user must be able to choose the option of deleting a selected appointment.}\\
Appointment Manager implements the functionality deleteAppointment().

\item \textbf{[R5] The system must be able to provide the user with an overview of his calendar and the user
must be able to view all appointments fixed in a certain period.}\\
The system gets information about all the created appointment from the database and display them to with the interfaces described in the User Interfaces section of this document.

\item \textbf{[R6] The user must be able to select a chosen day from the overview of his calendar.}\\
The system keeps track of the different dates (see Class Diagram: Date class) and offers the daily calendar option of display. See section~\ref{subsec:view}: Calendar views.

\item \textbf{[R7] The user must be able to select a specific appointment in his calendar.}\\
Account Manager loads the information about the different appointments (see Class Diagram: Appointment class). See section~\ref{subsec:editAppointment}: Editing existing appointment to see the correspondent user interfaces.

\item \textbf{[R8] The system must ask the user to provide all information needed for the creation of a new
appointment, such as place and time of start and overall duration.}\\
This requirement is covered by the createAppointment() functionality of Appointment Manager. See section~\ref{subsec:newAppointment}: Create new appointment.

\item \textbf{[R9] The system must check if the information provided by the user are correct.}\\
Functionalities covered during phases of creation/editing. See Sequence diagrams on section 5.2 of RASD document.

\item \textbf{[R10] The system must check if an appointment overlaps with other events and must eventually
notify it to the user.}\\
See section~\ref{subsec:algOverlap}: Check overlap algorithm.

\item \textbf{[R11] The system must give the user access to all details of a selected appointment and the user
must be allowed to edit the information needed.}\\
Account Manager loads the information about the different appointments (see Class Diagram: Appointment class) and display them to the user. It also implements the functionality editAppointment(). See section~\ref{subsec:editAppointment}: Editing existing appointment to see user interfaces.

\item \textbf{[R12] The user must be able to set advanced information for a created appointment.}\\
See advancedOptions() in section~\ref{subsec:newAppointment}: Create new appointment.

\item \textbf{[R13] The user must be able to set an appointment as flexible, specifying the interval of time.}\\
Functionality of Appointment Manager. See advancedOptions() in section~\ref{subsec:newAppointment}: Create new appointment and the Flexible Appointment class in the Class Diagram.

\item \textbf{[R14] The user must be able to set an appointment as repeatable, specifying the desired days.}\\
Functionality of Appointment Manager. See advancedOptions() in section~\ref{subsec:newAppointment}: Create new appointment.

\item \textbf{[R15] The system must schedule any flexible or repeatable appointment in the correct way, avoiding overlapping with other appointments.}\\
The system is provided with algorithm for overlapping checking and flexible appointments scheduling. See section~\ref{subsec:algorithm} for further info.

\item \textbf{[R16] The appointment intended to be modified must have been previously successfully created
and not already deleted.}\\
The process of editing appointments is made by Appointment Manager and works only with existing appointments in the database. A process is stored in the database only at the creation and removed during the deletion. See section~\ref{subsec:runtimeView}: Runtime view.

\item \textbf{[R17] The user must confirm the creation of the new appointment.}\\
createAppointment() functionality of Appointment Manager. See confirmCreation() in section~\ref{subsec:newAppointment}: Create new appointment.

\item \textbf{[R18] The user must confirm any appointment modification.}\\
editAppointment() functionality of Appointment Manager. See saveModification() in section~\ref{subsec:newAppointment}: Create new appointment.

\item \textbf{[R19] The system must save the user modifications in memory and the calendar must be updated.}\\
Appointment Manager interacts with the DBMS: after any modification, the database is updated. See section 2.4: Runtime view.

\item \textbf{[R20] The system must remove a deleted appointment from the memory and delete every alert
related to it.}
Appointment Manager interacts with the DBMS: after any modification, the database is updated.
This functionality is implemented by deleteAppointment().

\item \textbf{[R21] The user must be able to switch between different possible calendar, such as daily calendar, weekly calendar and monthly calendar.}\\
See section~\ref{subsec:view}: Calendar view.

\item \textbf{[R22] The system must be able to provide information about the scheduled travels for a chosen
day, showing the transport means and the estimated time required from each travel.}\\
User can recall the function viewDailySchedule() from his application, and Travel Manager takes care of provide all the information in detail. See section~\ref{subsec:runtimeView}: Runtime view.

\item \textbf{[R23] The system must choose the best option between the possible travel alternatives according
to the preferences expressed in the user profile settings and the information about external weather.}\\
This requirement is covered by the computeTravel() and loadPreferences() functionality of Travel Manager. See section~\ref{subsec:algComputeTravel}: Compute travel for further info on the algorithm.

\item \textbf{[R24] The user must be able to select a specific travel in his daily schedule.}\\
See selectTravel() in section~\ref{subsec:dailySchedule}: View daily schedule and travel/movement details.

\item \textbf{[R25] The system must provide detailed information about the travels selected by the user, such
as the trace route on the map and the weather conditions.}\\
This requirement is covered by the computeTravel() functionality of Travel Manager. More specifically, see section~\ref{subsec:runtimeView} for runtime view and section~\ref{subsec:algComputeTravel}: Compute travel for further info on the algorithm.

\item \textbf{[R26] The system must provide the user with an overview of the possible travel alternatives for the chosen travel, specifying all details for each one.}\\
Travel Manager functionalities. see section~\ref{subsec:algTravelAlternative}: Check travel alternative for further info on the algorithm and viewAlternatives() in section~\ref{subsec:dailySchedule}: View daily schedule and travel/movement details for user interfaces.

\item \textbf{[R27] The user must be able to filter the travel alternatives furnished.}\\
Travel Manager functionalities. see section~\ref{subsec:algTravelAlternative}: Check travel alternative for further info on the algorithm.

\item \textbf{[R28] The user must be able to choose a favorite travel option different from the displayed default one.}\\
Travel Manager functionalities. See section~\ref{subsec:algTravelAlternative}: Check travel alternatives for further info on the algorithm.

\item \textbf{[R29] The user must be able to select a specific movement in a travel.}\\
See selectMovement() in section~\ref{subsec:dailySchedule}: View daily schedule and travel/movement details.

\item \textbf{[R30] The system must provide detailed information about the movements selected by the user,
such as the specific trace route on the map and the price of the ticket.}\\
Travel Manager functionalities. All information are provided by the external APIs and collected by the system as explained in section~\ref{subsec:runtimeView}. See section~\ref{subsec:dailySchedule}: View daily schedule and travel/movement details for user interfaces.

\item \textbf{[R31] The user must be able to choose an alternative transport mean for a selected movement, if
there are any.}\\
Travel Manager functionalities. See section~\ref{subsec:algMovementAlternative}: Check movement alternative for further info on the algorithm.

\item \textbf{[R32] The system must update the daily schedule according to the travel option chosen by the user and the user must be able to see the new updated schedule.}\\
Travel Manager functionality. The user can choose a travel alternative (see section~\ref{subsec:algTravelAlternative}: Check travel alternative for info on the algorithm) and Travel Manager updates info.

\item \textbf{[R33] The system must give to the user the possibility of buying the ticket for the selected travel.}\\
This requirement is covered by Ticket Manager functionalities. See section~\ref{subsec:dailySchedule}: Component interfaces for full functionalities overview.

\item \textbf{[R34] The system must save a copy of the bought tickets.}\\
Tickets manager interacts with DMBS to store tickets data after the purchase. See section~\ref{subsec:runtimeView} for runtime views.

\item \textbf{[R35] The user must be able to access to a ticket page from the home page.}\\
Bought tickets are saved in the database during the purchase process (see section~\ref{subsec:runtimeView} for runtime view) and the user can access them by clicking on “My tickets) on the side panel menu. See section~\ref{subsec:ticket}: Buy and view tickets for user interfaces.

\item \textbf{[R36] The system must provide a list of all the bought tickets and the user must be able to select
and view a specific one.}\\
This requirement is covered by the viewTickets() functionality of Travel Manager. See section~\ref{subsec:ticket}: Buy and view tickets for user interfaces.

\item \textbf{[R37] The user must be able to access the preferences panel of his account.}\\
Account Manager functionality. See section~\ref{subsec:preferences}: Manage preferences for user interfaces.

\item \textbf{[R38] The system must give the user the possibility of setting various preferences, such as owned
and preferred travel means, address of Home and other general travel preferences.}\\
Account Manager functionality. See section~\ref{subsec:preferences}: Manage preferences for user interfaces.

\item \textbf{[R39] The user must be able to edit the provided preferences when needed.}\\
The preferences panel is always accessible from the home page (See section~\ref{subsec:preferences}: Manage preferences for user interfaces). Account Manager interacts with DBMS, so any modification will be update in the database.

\item \textbf{[R40] The system must give the user the possibility of adding an alert to an appointment while it is being created or modified.}\\
Functionality of Appointment Manager. See section 5.2.3 and 5.2.4 or RASD document: Appointment creation and Appointment editing for info on Sequence diagrams and sections~\ref{subsec:newAppointment}/~\ref{subsec:editAppointment} of this document for user interfaces

\item \textbf{[R41] The user must be able to choose a desired interval of time for the warning alert.}\\
Functionality of Appointment Manager. See section 5.2.6 of RASD document: Alert editing for info on Sequence diagrams and sections~\ref{subsec:newAppointment}/~\ref{subsec:editAppointment} of this document for user interfaces.

\item \textbf{[R42] The user must confirm the alert creation and the system must save the insertion in the memory.}\\
Appointment manager interacts with DBMS to store information in the system memory. See section 5.2.6 of RASD document: Alert editing for info on Sequence diagrams and sections~\ref{subsec:newAppointment}/~\ref{subsec:editAppointment} of this document for user interfaces.

\item \textbf{[R43] The user must be able to modify or remove the inserted alert when needed.}\\
Appointment manager functionalities are always accessible by the user through the correspondent interfaces.

\item \textbf{[R44] In case of any alert modification made by the user, the user must confirm the modification
and the system must save all changes.}\\
Appointment manager interacts with DBMS and updates data after any modification. See section 5.2.6 of RASD document: Alert editing for info on Sequence diagrams and sections~\ref{subsec:newAppointment}/~\ref{subsec:editAppointment} of this document for user interfaces.
\end{itemize}
