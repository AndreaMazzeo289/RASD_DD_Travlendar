\subsection{Scenario 1}
Andrea has just booked a last minute flight from Milan Orio al Serio airport to Prague, in Czech Republic, but being not a frequent flyer, he’s pretty worried about the idea of losing his plane. Furthermore, he lives far away from the airport and he needs to reach it by public transportation. Two days before his departure, he decides to download Travlendar+ app. He creates a new account by connetting his Facebook profile and fills out the essential account settings, giving his basic preferences (he doesn’t have any car or bike and he prefers cheaper travels). Then he creates a new appointment called “Prague”in his calendar, inserting the date of Saturday 11/11/17, the time 13:00 and the location of the airport. He chooses the option “I want to be there…” and select “2 hours before”. Then he creates the new appointment. His calendar now shows the “Prague” event, and the app easily displays how to reach the airport in time, by leaving home at 9:32, walking until the nearest metro station, taking the metro until Stazione Centrale and then taking a public bus to the airport at 10:10. Now Andrea feels much more confident!
\subsection{Scenario 2}
Serena has been using the Travlendar+ app since a couple of weeks. She has just bought a new car that allows her to move trough the city in a easier way, and wants the app to consider that when it display the optimal travel solution. She opens the app and moves to the preferences panel of her account. She then select “Travel means owned” and puts a tick in the “car” option. Then, she open her calendar and check her daily schedules for the next three days. Some of the travels suggested by the app are now changed and replaced with faster and more confortable movements with car.
\subsection{Scenario 3}
Marco has a scheduled appointment in program for the following day, and the app used to suggest a 7 minutes-movement by bike until the train station. But now the weather widget seems to announce a rainy day, and the app switched to a warmer travel by metro. But Marco is not afraid of rain and likes walking, so he opens the daily schedule, selects the metro movement and checks the possible travel alternatives. He then chooses the option “by walk”. The app nows shows the updated schedule, and the system will remember the choice for the future.
\subsection{Scenario 4}
Riccardo is a pretty absent-minded man, and has a morning full of appointments in schedule for the next day. He has added all the meetings in his calendar, but must absolutely not forget them, so he decides to add an alarm to each of them, in order to remember his commitments. He opens the app and select one by one all his appointments of Tuesday. For each one, he taps on “add an alert to this appointment” and chooses to be warned 15 minutes before the start. 