\subsection{Purpose}
This document represents the Requirement Analysis and Specification Document (RASD). The goal of this document is to describe the software application and focus on all its features. Furthermore, it’s interested to describe the functional and non-functional requirements of the system.
Show the constraint, imposed by stakeholders and application environment, the limits of the software.
This document is intended to all people that are interested to the project, such as stakeholders, investors and all developer and programmer that have to implement the application.

\subsection{Scope}
The application to develop is a mobile application that is called Travlendar+. This software is intended to help people with many commitments to manage the calendar on their smartphone. 
The only action that the user has to do is insert his daily appointments. The application should be able to organize the whole user’s day, providing advice and reminding all inserted appointment.
The application aims to be an advanced calendar management system, since it isn’t a simple appointments reminder but it has a lot functionality that allow to the user to be always well organized.
Lot are the functionality that the application provides, such as the complete transport management, that allow to compute the travel time and to identify the better travel solution basing on user’s preferences and environment information, such as weather conditions.
The user can choose if travel with own car or walk. He can decide to travel also in public transport and the application provides to the user the transport schedules and which transport choose. The system allows also the functionality to buy a ticket in-app.
Furthermore, the application is able to find the car sharing or bike sharing points nearest to the user.
It has an advices system when the appointment and the travel times overlaps. 
Daily the application can set a little time window (at most half an hour) reserved for the lunch. As this functionality, the user can schedule little break that the application set in day autonomously.

\subsection{Definition, Acronyms, Abbreviations}
\subsubsection{definition}
\begin{itemize}
	\item \textbf{Daily view}:  
	\item \textbf{Monthly view}:
	\item \textbf{Weekly view}:
	\item \textbf{Daily schedule}:
	\item \textbf{Appointment}: 
	\item \textbf{Alert}:
	\item \textbf{Travel}:
	\item \textbf{Movement}:
	\item \textbf{Unreachability}:
	\item \textbf{Overlapping}:
	\item \textbf{Green}:
	\item \textbf{Cheaper}:
	\item \textbf{Faster}:
	\item \textbf{Repeatable}:
	\item \textbf{Flexible}:
\end{itemize}
\subsubsection{acronyms}
\begin{itemize}
	\item \textbf{RASD}: Requirement Analysis and Specification Document.
	\item \textbf{API}: Application Programming Interface.
\end{itemize}
\subsubsection{abbreviations}

\subsection{Revision history}
\subsection{Reference documents}
\subsection{Document structure}
This RASD is composed by 5 parts and an appendix:
\begin{enumerate}
	\item The first part of RASD document is an introduction to the problem. The base information needed to understand the project scope are given in this section.
	\item The second part consists of an overall description of the system. Are described the characteristics that reguard the user, and all the application boundaries.
	\item The third part is composed by the specific requirements identified, both functional and non
	functional.
	\item In the fourth part a list of eight scenarios is provided; each of them describes a particular
	situation with the system might have to cope with.
	\item The fifth part is entirely composed by the UML diagrams that model the system in details.
	\item Sixth part is embodied with the Alloy model of the system and includes all the relevant
	details; a proof of consistency and an example of the generated world are also provided.
\end{enumerate}